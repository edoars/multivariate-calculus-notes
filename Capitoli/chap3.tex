%!TEX root = ../main.tex
%%%%%%%%%%%%%%%%%%%%%%%%%%%%%%%%%%%%%%%%%
%
%LEZIONE 30/03/2016 - SESTA SETTIMANA (1)
%
%%%%%%%%%%%%%%%%%%%%%%%%%%%%%%%%%%%%%%%%%
\chapter{Integrali dipendenti da parametro}
%%%%%%%%%%%%%
%CONTINUITA'%
%%%%%%%%%%%%%
\section{Continuità}

\begin{defn}{Integrale dipendente da parametro}{integraleParametro}\index{Integrale!dipendente da parametro}
	Sia \(f\colon \R^n \times \R^m \to \R,(x,y)\mapsto f(x,y)\), definiamo \emph{integrale dipendente dal parametro \(y\)} la funzione
	\[
		g\colon \R^m \to \R, y \mapsto \int\limits_{A(y)} f(x,y)\,\dd x.
	\]
\end{defn}

\begin{teor}{Continuità sotto segno di integrale}{continuitàIntegrale}
	Siano \(K\subseteq \R^n\) un compatto P-J misurabile e \(\Omega \subseteq \R^m\) aperto.
	Sia \(f\in C(K\times \Omega, \R)\), allora
	\[
		g\colon \Omega \to \R, y \mapsto \int\limits_K f(x,y)\,\dd x,
	\]
	è continua su \(\Omega\).
\end{teor}

\begin{proof}
	Sia \(y_0\in \Omega\), dobbiamo mostrare che \(g\) è continua in \(y_0\).
	Dal momento che \(\Omega\) è aperto posso prendere \(r>0\) tale che \(\overline{B_r(y_0)}\subseteq \Omega\).
	Allora \(f\colon K\times \overline{B_r(y_0)}\to \R\) è uniformemente continua, in quanto \(f\) è continua e \(K\times \overline{B_r(y_0)}\) è compatto.\graffito{il prodotto di compatti è compatto per il teorema di Tychonoff}
	Quindi, sfruttando una proprietà più debole dell'uniforme continuità, avremo
	\[
		\fa\e>0\,\ex\d>0 : \abs*{(x,y)-(x,y_0)}<\d \implies \abs*{f(x,y)-f(x,y_0)}<\e,\,\fa x\in K,y \in \overline{B_r(y_0)}.
	\]
	Da cui
	\[
		\begin{split}
			\abs*{g(y)-g(y_0)} & = \abs*{\int\limits_K f(x,y)\,\dd x-\int\limits_K f(x,y_0)\,\dd x}\\
			& \le \int\limits_K \abs*{f(x,y)-f(x,y_0)}\,\dd x\\
			& \le \int\limits_K \e\,\dd x \qquad\text{quando }d(y,y_0)\le\d,
		\end{split}
	\]
	quindi \(\abs*{g(y)-g(y_0)}\le \e\abs{K}\), ovvero \(g\) è continua.
\end{proof}

\begin{teor}{Continuità sotto segno di integrale (caso generale)}{continuitàIntegraleGenerale}
	Siano \(A \subseteq \R^n\), anche non limitato, P-J misurabile e \(\Omega \subseteq \R^m\) aperto.
	Sia \(f\in C(A\times \Omega, \R)\) e supponiamo che
	\[
		\abs*{f(x,y)} \le h(x),\,\fa x\in A,y \in \Omega,
	\]
	con \(h\colon A \to [0,+\infty)\) integrabile.
	Allora
	\[
		g\colon \Omega \to \R, y \mapsto \int\limits_A f(x,y)\,\dd x,
	\]
	è continua su \(\Omega\).
\end{teor}

\begin{proof}
	Dalla definizione di integrale improprio tramite estremo superiore sui compatti P-J misurabile, trovo \(K\subset A\) compatto e P-J misurabile tale che
	\[
		\int\limits_K h(x) \ge \int\limits_A h(x)\,\dd x - \frac{\e}{4},
	\]
	da cui
	\[
		\int\limits_{A\setminus K}h(x)\,\dd x = \int\limits_A h(x)\,\dd x - \int\limits_K h(x)\,\dd x \le \frac{\e}{4}.
	\]
	Sia \(y_0\in \Omega\), andiamo a mostrare la continuità di \(g\) in \(y_0\):
	\[
		\begin{split}
			\abs*{g(y)-g(y_0)} & = \abs*{\int\limits_A \big[f(x,y)-f(x,y_0)\big]\,\dd x}\\
			& \le \int\limits_A \abs*{f(x,y)-f(x,y_0)}\,\dd x\\
			& = \int\limits_K \abs*{f(x,y)-f(x,y_0)}\,\dd x + \int\limits_{A\setminus K}\abs*{f(x,y)-f(x,y_0)}\,\dd x.
		\end{split}
	\]
	Dove
	\[
		\int\limits_K \abs*{f(x,y)-f(x,y_0)}\,\dd x \le \frac{\e}{2}
	\]
	per il teorema precedente quando \(d(y,y_0)\le \d\).
	Mentre
	\[
		\begin{split}
			\int\limits_{A\setminus K}\abs*{f(x,y)-f(x,y_0)}\,\dd x & \le \int\limits_{A\setminus K}\abs*{f(x,y)}\,\dd x + \int\limits_{A\setminus K}\abs*{f(x,y_0)}\,\dd x\\
			& \le 2\int\limits_{A\setminus K}h(x)\,\dd x \le \frac{\e}{2}.
		\end{split}
	\]
	Ovvero
	\[
		\fa \e >0 \,\ex \d >0 : d(y,y_0)\le \d \implies \abs*{g(y)-g(y_0)} \le \e,
	\]
	quindi \(g\) è continua.
\end{proof}

\begin{oss}
	Se \(A\) fosse compatto, il maggiorante sarebbe il \(\sup f\).
\end{oss}

\begin{ese}[Integrale di Fresnel]
	Vogliamo calcolare
	\[
		\int_0^{+\infty} \frac{\sin x}{x}\,\dd x \overset{df}{=} \lim_{R\to +\infty} \int_0^R \frac{\sin x}{x}\,\dd x.
	\]
	Definiamo
	\[
		g(t) = \int_0^{+\infty} e^{-t\,x} \frac{\sin x}{x}\,\dd x.
	\]
	La strategia è trovare \(g(t)\) per \(t>0\), in tal caso avremo
	\[
		g(0) = \int_0^{+\infty} \frac{\sin x}{x}.
	\]
	Il problema è verificare la continuità in \(0\).
	Per farlo dobbiamo dimostrare che \(g\colon (0,+\infty)\to \R\) è continua tramite il teorema di continuità sotto segno di integrale (nel caso generale).
	Controlliamone le ipotesi:
	\begin{itemize}
		\item \(\displaystyle f\colon (0,+\infty)\times (0,+\infty) \to \R, (x,t)\mapsto e^{-t\,x}\frac{\sin x}{x}\) è banalmente continua.
		\item Nell'intorno di \(t_0>0\) trovo un maggiorante di \(f\) che sia integrabile e che non dipenda da \(t\)
		      \[
			      \abs*{e^{-t\,x} \frac{\sin x}{x}} \le e^{-\frac{t_0}{2}x} \qquad\text{per }t\in \left[ \frac{t_0}{2},+\infty \right),
		      \]
		      in quanto \(e^{-t\,x}\) è una funzione monotona decrescente e il rapporto \(\frac{\sin x}{x}\) si mantiene limitato al crescere di \(x\).
	\end{itemize}
	Abbiamo quindi dimostrato la continuità per \(t_0>0\), resta da verificare la continuità in \(0\).
	Affermo che
	\[
		\lim_{t\to 0} g(t) = \int_0^{+\infty} \frac{\sin x}{x}\,\dd x,
	\]
	infatti
	\[
		\begin{split}
			g(t) & = \int_0^{+\infty} e^{t\,x} \frac{\sin x}{x}\,\dd x = \lim_{R\to +\infty} \int_0^R e^{-t\,x} \frac{\sin x}{x}\,\dd x\graffito{integro \(\sin x\) per parti prendendo \(1\) come costante additiva}\\
			& = \lim_{R\to +\infty} \left[ \left.\frac{e^{-t\,x}}{x} (1-\cos x)\right|_0^R + \int_0^R \frac{x\,t+1}{x^2}e^{-t\,x}(1-\cos x)\,\dd x \right]\\
			& = \lim_{R\to +\infty} \int_0^R \frac{x\,t+1}{x^2}e^{-t\,x}(1-\cos x)\,\dd x = \int_0^{+\infty} \frac{x\,t+1}{x^2}e^{-t\,x}(1-\cos x)\,\dd x.
		\end{split}
	\]
	Ora
	\[
		(x,t) \mapsto \frac{x\,t +1}{x^2}e^{-t\,x}(1-\cos x),
	\]
	è continua su \((0,+\infty)\times (0,+\infty)\) in quanto
	\[
		\frac{x\,t +1}{x^2}e^{-t\,x}(1-\cos x) = e^{-t\,x}(x\,t+1) \frac{1-\cos x}{x^2} \to \frac{1}{2}.
	\]
	Cerchiamo il maggiorante integrabile per
	\[
		\abs*{\frac{x\,t+1}{x^2}e^{-t\,x}(1-\cos x)}.
	\]
	Consideriamo \(l(z)=(z+1)e^{-z}\), dal momento che \(l\) è continua e \(l(+\infty)=0\), per Weierstrass generalizzato esiste un massimo, ovvero \(\abs*{l(z)}\le M\).
	Quindi
	\[
		\abs*{\frac{x\,t+1}{x^2}e^{-t\, x}(1-\cos x)} \le M \frac{1-\cos x}{x^2},
	\]
	che è integrabile.
	Per cui \(g\) è continua in \(0\) e vale
	\[
		\lim_{t\to 0^+} g(t) = g(0) = \lim_{R\to +\infty}\int_0^R \frac{\sin x}{x}\,\dd x = \int_0^{+\infty} \frac{\sin x}{x}\,\dd x.
	\]
	Quindi se conoscessimo \(g(t)\) per \(t>0\), passando al limite otterremo l'integrale cercato.
	La conclusione di questo esempio verrà discussa nel prossimo paragrafo.
\end{ese}
%%%%%%%%%%%%%%%
%DERIVABILITA'%
%%%%%%%%%%%%%%%
\section{Derivabilità}

\begin{teor}{Derivata sotto segno di integrale}{derivataIntegrale}
	Siano \(K\subseteq \R^n\) un compatto P-J misurabile e \(\Omega \subseteq \R^m\) aperto.
	Sia \(f\in C(K\times \Omega,\R)\) tale che \(\pd_{y_i}f\in C(K\times \Omega,\R)\) per \(1\le i \le m\).
	Allora
	\[
		g\colon \Omega \to \R, y \mapsto \int\limits_K f(x,y)\,\dd x,
	\]
	è di classe \(C^1\) e vale
	\[
		g'(y) \cdot h = \int\limits_K \pd_y f(x,y) \cdot h\,\dd x.
	\]
\end{teor}

\begin{proof}
	Basta mostrare che la derivata è data dall'ultima formula, in tal caso sarà continua per il teorema di continuità sotto segno di integrale, in quanto \(K\) è compatto e \(\pd_y f\) è continua per ipotesi.
	La strategia sarà dimostrare la continuità di ogni componente di \(g'(y)\) per poi applicare il teorema del differenziale totale ed ottenere la continuità della derivata totale.

	Supponiamo quindi che \(\Omega \subseteq \R\). Per la definizione di derivata, dobbiamo mostrare che
	\[
		\abs*{g(y+h)-g(y)-\int\limits_K \pd_y f(x,y) h\,\dd x} = o(\abs{h}).
	\]
	Sfruttiamo il teorema del valore medio di Lagrange.
	Ricordiamo che in generale, presa una funzione \(h\colon U \to \R\) con \(U\subseteq \R^n\) e presi \(a,b \in U\) tali che il segmento che congiunge \(a,b\) è tutto contenuto in \(U\), vale
	\[
		h(b)-h(a) = h'\big(a+\q(b-a)\big)\cdot (b-a) \qquad\text{con }\q\in[0,1].
	\]
	Per cui, nel nostro caso monodimensionale,
	\[
		\begin{split}
			\abs*{g(y+h)-g(y)-\int\limits_K \pd_y f(x,y) h\,\dd x} & = \abs*{\int\limits_K f(x,y+h)\,\dd x-\int\limits_K f(x,y)\,\dd x-\int\limits_K \pd_y f(x,y) h\,\dd x}\\
			& \le \int\limits_K \abs*{f(x,y+h)-f(x,y)-\pd_y f(x,y) h}\,\dd x\\
			& = \int\limits_K \abs*{\pd_y f(x,y+\q\,h) h - \pd_y f(x,y) h}\,\dd x.
		\end{split}
	\]
	Sappiamo che \(\pd_y f\) è continua su \(K\times \Omega\) pertanto sarà uniformemente continua su \(K \times \overline{B_r(y)}\).
	Preso \(\abs{h}\le\d\) avremo
	\[
		\abs*{\pd_y f(x,y+\q\,h)h-\pd_y f(x,y)h} = \abs*{\pd_y f(x,y+\q\,h)-\pd_y f(x,y)}\abs{h} \le \e \abs{h}.
	\]
	Da cui
	\[
		\int\limits_K \abs*{\pd_y f(x,y+\q\,h) h - \pd_y f(x,y) h}\,\dd x \le \int\limits_K \e \abs{h}\,\dd x = \e \abs{K}\abs{h} = o\big(\abs{h}\big).\qedhere
	\]
\end{proof}

\begin{oss}
	Nell'espressione
	\[
		g'(y) \cdot h = \int\limits_K \pd_y f(x,y) \cdot h\,\dd x,
	\]
	abbiamo \(g'(y)\in (\R^m)^*, h\in R^m\) e
	\[
		\int\limits_K \pd_y f(x,y) \cdot h\,\dd x = \left( \int\limits_K \pd_{y_1}f(x,y)\,\dd x, \ldots, \int\limits_K \pd_{y_m} f(x,y)\,\dd x \right)
		\begin{pmatrix}
			h_1    \\
			\vdots \\
			h_m
		\end{pmatrix}
	\]
\end{oss}

\begin{teor}{Derivata sotto segno di integrale (caso generale)}{derivataIntegraleGenerale}
	Siano \(A \subseteq \R^n\), anche non limitato, P-J misurabile e \(\Omega \subseteq \R^m\) aperto.
	Sia \(f\in C(A\times \Omega, \R)\) e tale che \(\pd_{y_i}f(x,y)\in C(K \times \Omega, \R)\) per \(1\le i\le m\).
	Supponiamo che \(f(x,y)\) sia integrabile su \(A\) per ogni \(y\in \Omega\) e
	\[
		\abs*{\pd_{y_i}f(x,y)} \le h(x),\,\fa x \in A,y \in \Omega,\,\fa i \in (1,\ldots,m),
	\]
	con \(h\colon A \to [0,+\infty)\) integrabile.
	Allora
	\[
		g\colon \Omega \to \R, y \mapsto \int\limits_A f(x,y)\,\dd x,
	\]
	è di classe \(C^1\) e vale
	\[
		\pd_{y_i} g(y) = \int\limits_A \pd_{y_i} f(x,y)\,\dd x.
	\]
\end{teor}

\begin{proof}
	Non fornita.
\end{proof}

\begin{ese}
	Concludiamo l'esempio del paragrafo precedente cercando di capire chi è
	\[
		g(t)= \int_0^{+\infty} e^{-t\,x} \frac{\sin x}{x}\,\dd x.
	\]
	Affermo che \(g'(t)\) esiste per ogni \(t>0\). Verifichiamolo tramite le ipotesi del teorema:
	\begin{itemize}
		\item L'integrabilità di \(f\) è già stata verificata nella parte precedente dell'esempio.
		\item Dobbiamo trovare la maggiorazione uniforme di \(\pd_t f\):
		      \[
			      \abs*{\pd_t \left( e^{-t\,x} \frac{\sin x}{x} \right)} = \abs*{e^{-t\,x}\sin x} \le e^{-\frac{t_0}{2}x} \qquad\text{per }t\in \left[\frac{t_0}{2},+\infty\right).
		      \]
	\end{itemize}
	Quindi \(g\) è di classe \(C^1\) e vale
	\[
		g'(t) = -\int_0^{+\infty} e^{-t\,x} \sin x\,\dd x,
	\]
	il quale, svolto per parti, trova
	\[
		g'(t) = -\frac{1}{1+t^2},
	\]
	ovvero \(g(t)= c-\arctan t\).
	Da \(g(+\infty)= 0\) segue \(g(t)=\frac{\p}{2}-\arctan t\), quindi
	\[
		\int_0^{+\infty} \frac{\sin x}{x}\,\dd x = g(0) = \frac{\p}{2}.
	\]
\end{ese}
%%%%%%%%%%%%%%%%%%%%%%%%%%%%%%%%%%%%%%%%%%%
%
%LEZIONE 05/04/2016 - SETTIMA SETTIMANA (1)
%
%%%%%%%%%%%%%%%%%%%%%%%%%%%%%%%%%%%%%%%%%%%
\begin{ese}
	Si calcoli
	\[
		F(x) = \int_x^{x^2} e^{-(x-t)^2}\,\dd t.
	\]
	Osserviamo che in questo caso il dominio varia con \(x\).
	Introduciamo
	\[
		G(a,b,x) = \int_a^b e^{-(x-t)^2}\,\dd t,
	\]
	avremo quindi \(F(x)=G(x,x^2,x)\).
	Riassumendo \(F\) risulta la composizione
	\[
		\R \to \R^3 \to \R, x\mapsto (x,x^2,x) \mapsto G(x,x^2,x).
	\]
	In particolare
	\[
		F'(x) = G'(x,x^2,x) \begin{pmatrix}
			1  \\
			2x \\
			1
		\end{pmatrix}
	\]
	dove
	\[
		\begin{split}
			G'(x,x^2,x) & = \big( \pd_a G|_{(x,x^2,x)}, \pd_b G|_{(x,x^2,x)}, \pd_x G|_{(x,x^2,x)}\big)\\
			& = \left(-e^{-(x-a)^2}|_{(x,x^2,x)}, e^{-(x-b)^2}|_{(x,x^2,x)}, \int_a^b -2(x-t) e^{-(x-t)^2}\,\dd t|_{(x,x^2,x)} \right)\\
			& = \left(-1,e^{-(x-x^2)^2},\int_x^{x^2} -2(x-t) e^{-(x-t)^2}\,\dd t\right) 	\begin{pmatrix}
				1  \\
				2x \\
				1
			\end{pmatrix}
		\end{split}
	\]
	da cui è possibile calcolare la primitiva.
\end{ese}